\subsection{Project planning}

The project descriptions are generally significantly less detailed than what was made available for the checkpoints. Any material covered during checkpoints including python code examples are assumed to be known.  Only essential and new information is provided and you are expected to take care of the details. Python code snippets are provided where necessary, but you will have to understand yourself what they do. It is recommend that you google for information about your project on the web, including data sheets of components and python libraries, if applicable. Python scripts should be well structured, either using functions or classes.

The timeline will vary between different projects, but in general, it is recommended that you plan your work as follows:
\begin{itemize}
\item	weeks 7, 8 \& 9: 	Building your gadget and/or writing code for project;
\item	week 9, 10: 	Analysis of data or equivalent, prepare supplementary material;
\item	week 10, 11:	Finish writing of project report and prepare submission.
\end{itemize}
Note that you are advised to start writing your report as the project progresses. 

For guidance on report writing, how the projects will be assessed, plagiarism and the submission deadline, please consult the DAH course booklet and the DAH grade descriptors, available on Learn.

